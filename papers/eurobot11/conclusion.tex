In this paper we presented the design and current development of ERBPI, an application to provide a simple, graphical, behaviour-based interface for Educational Robotics courses aimed at inexperienced students. To reach this goal, we propose to abandon the imperative paradigm and take a behaviour based approach. The tool is capable of controlling different robotics platforms, and it is designed to be easily expandable for new ones. It is also portable to different operating systems to accommodate the available hardware and software in schools. Experiences with high school students show that the current version of ERBPI allows inexperienced public to quickly start working with robotic environments. 

We are currently working on the subsumption feature of the tool, important to allow the development of more complex behaviours. We also plan to expand the application in order to execute the behaviour directly on the robot when possible. Another improvement we are working on is to include a play-back facility to debug behaviours, using the execution log-file and a web-cam that films the robot behaviour. The idea is to use the execution log-file to show at the same time which sensors, connections and behaviours were activated when the robot took a certain action, and the play-back of the video captured with the web-cam. 

During the present year we will prepare and teach educational robotics courses for more high schools in Buenos Aires, using the full featured ERBPI tool. We hope this experiences will provide extra feedback to continue and improve this project. 

