%ERBPI's goal is to provide an easy interface that allows inexperienced students to interact with robots, thus providing a tool for Educational Robotics courses in schools. 
To explore the adequacy of the tool, a preliminary version of ERBPI was used in a special course designed for high school students during July 2010. 

The class was composed of 20 students from seven %technical (Comentario Pablo: yo sacaría que fueron escuelas técnicas porque toda la justificación era que la aplicación estaba dirigida a estudiantes que no sabían programar) 
high schools of Buenos Aires, Argentina. The student's programming experience varied from none to some experience with procedural languages like C, C++ or Pascal. There were also few students with some electronics background, and all mastered basic mathematical concepts such as constant, monotonous and broken functions. None of the students had experience with robots. 

The course covered basic concepts of robotics and of behaviour based robotics. The preliminary version of ERBPI used for the course only had the basic conectionnist approach, lacking the subsumption feature to compose more complex behaviours. We followed a hands-on approach, programming easy behaviours from day one in groups of two or three students. Using the Braitenberg model \cite{braitenberg}, the students developed behaviours in simulators that were also tested in real robots, experiencing with the Yaks simulator \cite{yaks}, Khepera \cite{khepera} and ExaBot \cite{exabot} robots. 
Different behaviours were proposed for the students to solve with the robots, ranging from follow a line in the floor, go to a light spot, avoid obstacles or solve mazes.
Those behaviours were developed by teaching students the scientific method, encouraging them to propose hypothesis, contrast the expected results with the ones obtained in the testing phase, and then propose explanations and changes to the original robot control. Each group also shared their findings with the rest of the class, showing different approaches and solutions to the given problems during a discussion phase. Overall, the students picked up the use of ERBPI easily and could program all the proposed behaviours quickly. %They also applied the base of the scientific method in their experiences naturally. 

After the course, a questionnaire was given to the students in order to explore their reviews of the course and the programming interface. All students answered that they had found the interface easy to learn and use, despite they had no idea about programming robots and had not had the opportunity to control a robot before the course. All students felt that they met the objectives of the course successfully and most of them showed interest in taking more courses of robotics, computer science and engineering after the course. 

In this experience we found some improvements to be made in ERBPI. For example, we plan to execute the resulting program directly in the robot when possible, instead of executing the program in the external PC with the consequent transmission of sensor data and commands back and forth. Another improvement is to increase the amount of possible functions, to broaden the tools capabilities for math teachers. We also realized how important the subsumption architecture feature is, to allow the construction of more complex behaviours from simpler ones. Students also suggested some improvements, like ``cooler'' or friendlier names for different objects of ERBPI. %(i.e. to call the constant speed functions for the motors ``power balls''). 

We also tested ERBPI in shorter courses, lectures, exhibitions and others activities of popularization of science for high school students and broader public. The most important was part of a three day exhibition of the University of Buenos Aires that took place in October 2010 \cite{expouba}. Many of the ideas in ERBPI were born in previous experiences with high school students, mainly two eight-week workshop on robotics we organized as a part of a program from the Vocational Orientation Department of our Faculty during 2006 \cite{taller2006} and 2009 \cite{taller2009}. 

We are planning on using a full featured ERBPI in Educational Robotics courses for more high schools of Buenos Aires during the present year, as a part of a popularization of science project of our Faculty. 