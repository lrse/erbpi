%Idea de esta secci�n: comparaci�n con las herramientas ya existentes. Esta bastante bien descripto en el informe final. 
%There are several available robot programming interfaces, both commercial and free, and also several papers on programming interfaces for Educational Robotics. In the commercial brand, Microsoft offers a 

Most robotics programming interfaces are designed for university-level or late high school students and are implemented as extensions to existing programming languages. That is the case of Pyro[cita], a Python-based programming framework which provides a set of abstractions that allow students to write platform-independent robot programs. Other interfaces include Not-Quite C (NQC) [1] based on C, BrickOS [14] based on C++ and leJOS [18] that is based on Java. All these interfaces require programming experience or interest in learning a particular programming language. This makes them unsuitable for the use in middle and high school educational robotics courses which aim is teaching subjects other than robotics or programming. 

Microsoft also offers the commercial tool Microsoft� Robotics Developer Studio. This includes a visual programming interface based on a data flow approach that requires knowledge of programming concepts such as conditional loops, ifs, variables, etc, making it quite complex for unexperienced public. 

There are also several graphical environments for simulated robots aimed at middle and high schools. That is the case of StartLogo[cita], Squeak Etoys[cita] and Scratch[cita]. These are easy-to-use programming interfaces, allowing unexperienced students to make a quick start, although they mantain an imperative programming influence and are designed only for particular simulated environmets. Probably the most well-known programming interface used in instructional settings at the K-12 level is RoboLab [22] for the LEGO Mindstorms robot . This is a graphical environment in which students are given �palettes� of �icons� that they can drag and drop on a canvas. The icons represent robot components like motors and sensors, as well as abstract programming structures such as loops and counter variables. This interface is particular for the Mindstrom robot, and once again uses programming structures that add complexity to the robotic environment. Finally, in [cita] authors present an extension to RoboLab in order to work with other robotic platforms.

RoboLab - Lego.

Simuladores (Scratch, StartLogo, Etoys)


Pyro - Programacion imperativa (Python) pero se puede usar con muchos robots diferentes.
Improve del RoboLab para Aibo.
Improve del RoboLab para 3 plataformas diferentes 

More recently, Blank et al. [3] introduced Pyro, a Python-based programming framework which provides a set of abstractions that allow students to write platform-independent robot programs. This versatile programming environment has been successfully integrated into a wide variety of existing computer science courses, from introductory programming to advanced mobile robotics courses. One of the unique features of Pyro is the write once/run anywhere (on any robot!) approach; whereas most robot programming interfaces tend to be specific to particular robotic platforms.

Most robotics programming interfaces are designed for university-level or late high school students and are implemented as extensions to existing languages. For example, Pyro is based on Python, Not-Quite C (NQC) [1] is based on C, BrickOS [14] is based on C++ and leJOS [18] is based on Java. There are fewer interfaces for students who lack programming experience or interest in learning a programming language. Probably the most well-known programming interface used in instructional settings at the K-12 level for the LEGO Mindstorms robot is RoboLab [22]. This is a graphical environment in which students are given �palettes� of �icons� that they can drag and drop on a canvas. The icons represent robot components like motors and sensors,
as well as abstract programming structures such as loops and counter variables. Figure 1 contains a sample RoboLab program. 

