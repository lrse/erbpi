%%%%%%%%%%%%%%%%%%%%%%%%%%%%%%%%%%%%%%%%%%%%%%%%%%%%%%%%%%%%
%% LOG
%%%%%%%%%%%%%%%%%%%%%%%%%%%%%%%%%%%%%%%%%%%%%%%%%%%%%%%%%%%%


\section{Introducci\'on}

El archivo de \textit{log} de cada ejecuci�n se encarga de esribirlo el \textit{Core}.
Siempre sobreescribe el archivo especificado, o lo crea si no existe, es decir, s�lo queda en el archivo el contenido de la �ltima ejecuci�n.

El \textit{log} tiene la siguiente especificaci�n:
\begin{enumerate}
 \item \textbf{La primera l�nea.} Consiste de la secuencia, separada por comas, de los \textit{ids} de la tabla de ejecuci�n en el orden en que se encuentran en la misma.
 \item \textbf{Siguientes l�neas.} Son todas iguales. Consiste de varios valores, separados por comas, de la siguiente forma:
	\begin{itemize}
	\item \textit{TimeStamp}. Es el tiempo en \textit{milisegundos} para cada l�nea relativo al comienzo, es decir, comienza en cero.
	\item \textit{Valor de los elementos}. En el mismo orden en que fueron detallados en la primera l�nea, 
		  si es una \texttt{caja} son los valores \textit{entrada} y \textit{salida} de la caja, y si es un \textit{sensor} o un \textit{actuador} es simplemente el valor de \textit{salida}.
	\end{itemize}
\end{enumerate}

Por ejemplo, el siguiente archivo de \textit{log} corresponde a 10 ejecuciones del \textit{Core}:
\footnotesize	% esto hace que el verbatim se vea chiquitito
\begin{verbatim}
        sonar.0, sonar.1, sonar.2, actuador.0, caja.0, caja.1, actuador.1, 
        0, 6, 8, 7, 8, 22, 13, 8, -9, 20, 
        103, 3, 7, 6, 7, 17, 13, 7, -8, 19, 
        204, 8, 3, 2, 3, 14, 13, 3, -6, 15, 
        304, 7, 4, 10, 4, 15, 13, 4, -6, 23, 
        405, 7, 3, 7, 3, 13, 13, 3, -6, 20, 
        505, 10, 8, 9, 8, 26, 13, 8, -9, 22, 
        606, 4, 3, 1, 3, 10, 13, 3, -6, 14, 
        707, 3, 10, 6, 10, 23, 13, 10, -13, 19, 
        807, 8, 10, 9, 10, 28, 13, 10, -13, 22, 
        1009, 3, 2, 7, 2, 7, 8, 2, -6, 15, 
\end{verbatim}
\normalsize	% esto termina el verbatim se vea chiquitito


\textbf{PENDIENTE}

REPLAY Y DEBUG...


\section{Core Implementaci\'on}


